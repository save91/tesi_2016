\chapter{Installazione Proximity System}
\section{Backend e WebApp}
In questa sezione vedremo come installare e avviare Proximity System.
Iniziamo con il backend scaricando il codice da GitHub
\begin{lstlisting}[style=JavaScriptCode]
git clone https://github.com/e-xtrategy/unicam-beacon-server.git
\end{lstlisting}
Spostiamoci dentro la cartella webserver e installiamo Node tramite NVM
\begin{lstlisting}[style=JavaScriptCode]
cd unicam-beacon-server/webserver
nvm install
\end{lstlisting}
Installiamo i moduli per il backend tramite npm e per la webapp tramite bower
\begin{lstlisting}[style=JavaScriptCode]
npm install
npm install -g bower
cd angular
bower install
cd ..
\end{lstlisting}
Assicuriamoci che il demone di MongoDB sia in esecuzione e lanciamo il seguente comando per popolare il database con i dati iniziali
\begin{lstlisting}[style=JavaScriptCode]
node test/config/setup_tests.js
\end{lstlisting}
Prima di far partire il webserver dobbiamo settare una variabile d'ambiente per far capire a Node se ci troviamo in un PC o in un RaspberryPI
\begin{lstlisting}[style=JavaScriptCode]
export NODE_ENV=development
\end{lstlisting}
Development indica che ci troviamo in fase di sviluppo e quindi su un PC.
Se ci trovassimo su un RaspberryPi potremmo settarla su production o, più semplicemente, non settarla.
Ora avviamo il webserver con il comando
\begin{lstlisting}[style=JavaScriptCode]
npm start
\end{lstlisting}
Se ci troviamo in un PC e non vogliamo eseguire il webserver con i permessi di root è anche possibile avviare il tutto tramite
\begin{lstlisting}[style=JavaScriptCode]
node server
\end{lstlisting}
Ora il nostro backend resterà in ascolto sulla porta 8000.
Attenzione al fatto che il codice funziona soltanto sui sistemi operativi basati su Debian.
Per far girare il programma su altri OS come Windows o Mac dobbiamo rimuovere il modulo rpi-gpio con npm 
\begin{lstlisting}[style=JavaScriptCode]
npm remove rpi-gpio
\end{lstlisting}
e commentare la prima riga di codice di
\begin{lstlisting}[style=JavaScriptCode]
/unicam-beacon-server/webserver/app/services/gpio.js
\end{lstlisting}
\section{App}
Passiamo all'app.
Iniziamo scaricando il codice dal repository su GitHub
\begin{lstlisting}[style=JavaScriptCode]
cd
git clone https://github.com/e-xtrategy/unicam-ionic-beacon-app.git
\end{lstlisting}
Assicuriamoci di utilizzare la giusta versione di Node
\begin{lstlisting}[style=JavaScriptCode]
cd app
nvm install
\end{lstlisting} 
Installiamo Ionic, Cordova, Bower e tutte le dipendenze del progetto
\begin{lstlisting}[style=JavaScriptCode]
npm install -g ionic cordova bower
bower install
ionic state restore
\end{lstlisting}
Passiamo all'esecuzione dell'app.
Per fare ciò è necessario avere l'ambiente di sviluppo configurato correttamente. 
Per Android, quindi, sarà necessario aver installato correttamente Android Studio o più generalmente l'SDK, mentre per iOS Xcode.
Adesso possiamo lanciare rispettivamente i seguenti comandi per android e iOS
\begin{lstlisting}[style=JavaScriptCode]
ionic run android
ionic emulate ios
\end{lstlisting}  

